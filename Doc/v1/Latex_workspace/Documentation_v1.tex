\documentclass[12pt]{article}

% Package
\usepackage{graphicx} % Add images package
\usepackage{listings} % Add code file package

% Document settings
\graphicspath{{../images}}
\setlength{\parindent}{0pt}

% Documentation title
\title{Sommatore a 3 input (v1)}
\author{Stefano Scarcelli \& Michele De Fusco}
\date{01 Dic 2023}


% Document
\begin{document}
% Cover
\maketitle
\newpage

% Indice
\tableofcontents
\newpage

% Document body

\section{Analisi progettuale}
    \subsection{Analisi preliminare}
        L'obbiettivo è quello di costruire un circuito in grado di sommare 3 numeri a \textit{n-bit} (\textit{2's complements}) e restituirne il risultato. Sia gli input che gli output devono essere sincronizzati tramite l'uso di registri.

        L'idea di base è quella di usare due \textbf{Ripple carry} \textit{parametrici} in cascata tra di loro per eseguire il calcolo desiderato $A+B+C=R$.
    
    \subsection{Struttura del progetto}
        L'idea alla base dell'implementazione è quella di inserire in pipeline i due moduli \textbf{sommatori} per eseguire prima la somma $A+B$ e successivamente $(A+B)+C$.

        Questa implementazione porta all'inserimento di (in totale) di 6 \textbf{registri}, 2 in ingresso ad ogni adder, uno in aggiunta all'input $C$ per salvare il risultato per la seconda operazione di somma e uno in uscita per salvare il risultato dell'operazione complessiva (richiesto dal progetto).

        Il primo ritardo per riceve un risultato coerente è di 2 colpi di clock mentre il delay per ricevere i risultati successivi al primo è di solo 1 colpo di clock.

\section{Implementazione}
    L'implementazione si basa sulla definizione di una struttura gerarchica di componenti, partendo dalla definizione comportamentale dei componenti elementari (\textbf{Adder n-bit} e \textbf{Register n-bit}) per poi andare a comporre (tramite 2 livelli di astrazione, \textbf{Synched adder} e in fine l'elemento principale \textbf{Three adder (main)}) la struttura del progetto.

    \subsection{Adder n-bit}
        L'implementazione dell'\textbf{Adder n-bit} segue la descrizione comportamentale tramite \textbf{Ripple carry} parametrico usando i segnali \textit{propagate} ($P$) e \textit{generate} ($G$).

        \subsubsection{Codice VHDL}
            % Codice VHDL
            
        \subsubsection{Sintesi}
            % Schemi sintetizzati

    \subsection{Register n-bit}
        L'implementazione dell'\textbf{Register n-bit} segue la descrizione comportamentale classica con memorizzazione a \textit{fronti di salita}.
        
        Il \textbf{registro} implementa in più un segnale di \textit{clear} (asincrono) attivo alto.

        \subsubsection{Codice VHDL}
            % Codice VHDL
            
        \subsubsection{Sintesi}
            % Schemi sintetizzati

    \subsection{Synched adder}
        % ...

        \subsubsection{Codice VHDL}
            % Codice VHDL
        
        \subsubsection{Sintesi}
            % Schemi sintetizzati

    \subsection{Three adder (main)}
        % ...

        \subsubsection{Codice VHDL}
            % Codice VHDL
        
        \subsubsection{Sintesi}
            % Schemi sintetizzati

\section{Testing}
    % ...

\end{document}