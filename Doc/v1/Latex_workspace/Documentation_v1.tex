\documentclass[12pt]{article}

% Package
\usepackage{graphicx} % Add images package
\usepackage{listings} % Add code file package

% Document settings
\graphicspath{{../images}}
\setlength{\parindent}{0pt}

% Documentation title
\title{Sommatore a 3 input (v1)}
\author{Stefano Scarcelli \& Michele De Fusco}
\date{01 Dic 2023}


% Document
\begin{document}
% Cover
\maketitle
\newpage

% Indice
\tableofcontents
\newpage

% Document body

\section{Analisi progettuale}
    \subsection{Analisi preliminare}
        L'obbiettivo è quello di costruire un circuito in grado di sommare 3 numeri a \textit{n-bit} (\textit{2's complements}) e restituirne il risultato. Sia gli input che gli output devono essere sincronizzati tramite l'uso di registri.

        L'idea di base è quella di usare due \textbf{Ripple carry} \textit{parametrici} in cascata tra di loro per eseguire il calcolo desiderato $A+B+C=R$.

\section{Implementazione}
    % ...

\section{Testing}
    % ...

\end{document}